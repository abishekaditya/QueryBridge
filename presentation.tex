
%%%%%%%%%%%%%%%%%%%%%%%%%%%%%%%%%%%%%%%%%%%%%%%%%%%%%%%%%%%%%%%
% GraphQL to Datalog Optimizer (Beamer deck)
% CSE505 Spring 2025 — Abishek Aditya
%%%%%%%%%%%%%%%%%%%%%%%%%%%%%%%%%%%%%%%%%%%%%%%%%%%%%%%%%%%%%%%

\documentclass{beamer}

% --- Packages --------------------------------------------------------
\usepackage{graphicx}
\usepackage{listings}
\usepackage{booktabs}
\usepackage{tikz}
\usepackage{xcolor}

% --- Listing colours -------------------------------------------------
\definecolor{graphql}{HTML}{E535AB}
\definecolor{datalog}{HTML}{1E90FF}

% --- Listings setup --------------------------------------------------
\lstset{
  basicstyle=\ttfamily\footnotesize,
  keywordstyle=\color{graphql},
  commentstyle=\itshape\color{gray},
  stringstyle=\color{datalog},
  showstringspaces=false,
  columns=fullflexible,
  frame=single,
  breaklines=true,
  tabsize=2
}

% --- Metadata --------------------------------------------------------
\title{GraphQL to Datalog Optimizer}
\author{Abishek Aditya}
\institute{CSE505 Spring 2025}
\date{\today}

% --------------------------------------------------------------------
\begin{document}

% 1 — Title -----------------------------------------------------------
\maketitle

% 2 — Outline ---------------------------------------------------------
\begin{frame}{Roadmap}
  \tableofcontents[hideallsubsections]
\end{frame}

\begin{frame}{Why am I here?}
        Alphabetical luck.
\end{frame}

% 3 — Motivation / Problem -------------------------------------------
\section{Motivation}
\begin{frame}{Motivation: Why optimize GraphQL?}
  \begin{itemize}
    \item Over-fetching leads to wasted bandwidth and latency
    \item Complex nested queries strain backend resources
    \item Deep queries incur exponential overhead with naive approaches
    \item Need for declarative, optimizable representation $\rightarrow$ \alert{Datalog}
    \item GraphQL lacks standard optimization techniques across various implementations
  \end{itemize}
  
  \begin{block}{Example Challenges}
    \begin{itemize}
      \item Nested queries retrieving all records before filtering
      \item Join order optimization not available across all implementations
      \item Missing opportunities for pushing filters down
    \end{itemize}
  \end{block}
\end{frame}

% 4 — Background ------------------------------------------------------
\section{Background}
\begin{frame}{Background: GraphQL \& Datalog}
  \begin{columns}[T]
    \begin{column}{0.48\textwidth}
      \textbf{GraphQL}
      \begin{itemize}
        \item Hierarchical selection of nested data
        \item Client-driven shape and field specification
        \item Popular API query language
        \item Built for graph-like data retrieval
        \item Lacks standardized backend optimization
      \end{itemize}
    \end{column}
    \begin{column}{0.48\textwidth}
      \textbf{Datalog}
      \begin{itemize}
        \item Declarative logic programming language
        \item Deep theoretical optimization foundation
        \item Suited for fixed-point analysis \& optimization
        \item Well-understood complexity guarantees
        \item Widely used in database theory and systems
      \end{itemize}
    \end{column}
  \end{columns}
\end{frame}

% 5 — System Architecture --------------------------------------------
\section{Architecture}
\begin{frame}[fragile]{Architecture: Project Structure}
  \begin{columns}[T]
    \begin{column}{0.48\textwidth}
      \textbf{Source Organization}
      \begin{lstlisting}[basicstyle=\tiny\ttfamily]
.
├── docs/
│   ├── demand_optimization_example.md
│   ├── demand_transformation.md
│   └── index.md
├── src/
│   └── querybridge/
│       ├── __init__.py
│       ├── __main__.py
│       └── translator.py
├── tests/
│   ├── basic/
│   ├── nested/
│   ├── complex_path/
│   └── path_finding/
├── run-tests.py
└── setup.py
      \end{lstlisting}
    \end{column}
    \begin{column}{0.48\textwidth}
      \textbf{Processing Pipeline}
      \begin{enumerate}
        \item \textbf{Parse} GraphQL schema and query $\rightarrow$ structured representation
        \item \textbf{Translate} to intermediate Datalog representation 
        \item \textbf{Optimize} via demand transformation (magic sets)
        \item \textbf{Generate} executable XSB Datalog code
      \end{enumerate}
      
      \textbf{Key Components}
      \begin{itemize}
        \item Core translator (\texttt{translator.py})
        \item Data structures for schema and query
        \item CLI for direct invocation
        \item Test suite with varied complexity
      \end{itemize}
    \end{column}
  \end{columns}
  
  \begin{block}{Module Dependencies}
    \begin{itemize}
      \item \texttt{graphql-core}: For parsing GraphQL schema and queries
      \item \texttt{xsb}: Runtime for executing generated Datalog queries
      \item Built with standard Python libraries for extensibility
    \end{itemize}
  \end{block}
\end{frame}

% 6 — Translation -----------------------------------------------------
\section{Translation}
\begin{frame}[fragile]{Translation Process}
  \begin{itemize}
    \item Parse GraphQL schema to understand types and relationships
    \item Parse GraphQL query to extract field selection and arguments
    \item Generate fact predicates from schema entities
    \item Generate rule predicates for query fields
    \item Apply demand transformation for optimization
    \item Generate final answer predicate
  \end{itemize}
  
  \begin{block}{Key Insights}
    \item Field selection becomes projection in Datalog
    \item Nested fields become join operations
    \item Arguments become value constraints
    \item Field paths translate to variable sharing
  \end{block}
\end{frame}

\begin{frame}[fragile]{Translation — GraphQL Query Example}
\begin{lstlisting}
{
  project(name: "GraphQL") {
    tagline
    contributors {
      name
      email
    }
  }
}
\end{lstlisting}
\end{frame}

\begin{frame}[fragile]{Translation — XSB Datalog (before optimization)}
\begin{lstlisting}[basicstyle=\scriptsize\ttfamily]
% Rules for field: project
project_result(ROOT) :- project_ext(ROOT), NAME = "GraphQL".
project_tagline_result(PROJECT_1, TAGLINE_2) :- tagline_ext(PROJECT_1, TAGLINE_2).
project_contributors_result(PROJECT_1) :- contributors_ext(PROJECT_1).
project_contributors_name_result(CONTRIBUTORS_3, NAME_4) :- name_ext(CONTRIBUTORS_3, NAME_4).
project_contributors_email_result(CONTRIBUTORS_3, EMAIL_5) :- email_ext(CONTRIBUTORS_3, EMAIL_5).

% Final answer predicate 
ans(TAGLINE, CONTRIBUTORS_NAME, CONTRIBUTORS_EMAIL) :- 
  project_ext(PROJECT_1), project_result(ROOT),
  project_tagline_result(PROJECT_1, TAGLINE), 
  project_contributors_result(PROJECT_1),
  project_contributors_name_result(CONTRIBUTORS_3, CONTRIBUTORS_NAME),
  project_contributors_email_result(CONTRIBUTORS_3, CONTRIBUTORS_EMAIL).
\end{lstlisting}
\end{frame}

\begin{frame}[fragile]{Translation — After Demand Transformation}
\begin{lstlisting}[basicstyle=\scriptsize\ttfamily]
% Demand transformation facts & rules
demand_project_B("GraphQL").
m_project_B(ROOT) :- demand_project_B("GraphQL").

% Propagate demand to contributors
demand_contributors__(PROJECT_1) :- m_project_B(ROOT), project_ext(PROJECT_1).
m_contributors__(PROJECT_1) :- demand_contributors__(PROJECT_1).

% Optimized rules
project_result(ROOT) :- m_project_B(ROOT), project_ext(ROOT), NAME = "GraphQL".
project_tagline_result(PROJECT_1, TAGLINE_2) :- tagline_ext(PROJECT_1, TAGLINE_2).
project_contributors_result(PROJECT_1) :- m_contributors__(PROJECT_1), contributors_ext(PROJECT_1).

% Nested fields have demand propagated to them
project_contributors_name_result(CONTRIBUTORS_3, NAME_4) :- 
  m_contributors__(PROJECT_1), contributors_ext(PROJECT_1, CONTRIBUTORS_3), 
  name_ext(CONTRIBUTORS_3, NAME_4).

ans(TAGLINE, CONTRIBUTORS_NAME, CONTRIBUTORS_EMAIL) :- ...
\end{lstlisting}
\end{frame}

% 7 — Demand Transformation ------------------------------------------
\section{Demand Transformation}
\begin{frame}{Demand Transformation: Big Idea}
  \begin{itemize}
    \item Magic Sets technique from database query optimization
    \item Derive \emph{only the facts your query needs}
    \item Add \textit{demand} ("magic") predicates as guards on computation
    \item Converts bottom-up evaluation to be query-driven (like top-down)
    \item Saves time and memory on large graphs or APIs
    \item Most beneficial for:
      \begin{itemize}
        \item Queries with selective filters
        \item Deeply nested relationships
        \item Large datasets
      \end{itemize}
  \end{itemize}
\end{frame}

\begin{frame}{Implementation in QueryBridge}
  \begin{itemize}
    \item Identify bound arguments (constraints/literals in query)
    \item Create adornment patterns (B for bound, F for free)
    \item Generate demand predicates as computation seeds
    \item Create magic predicates as rule guards
    \item Propagate demand through nested relationships
    \item Optimize predicates with bound arguments first
  \end{itemize}
  
  \begin{block}{Key Functions}
    \item \texttt{generate\_demand\_transformation()} - Creates demand and magic predicates
    \item Tracks applied transformations and reasons in logs
    \item Handles nested field propagation automatically
  \end{block}
\end{frame}

\begin{frame}[fragile]{Example — GraphQL with Filters}
\begin{lstlisting}
{
  users(minAge: 25, maxAge: 40, nameContains: "Smith") {
    name
    email
  }
}
\end{lstlisting}

\begin{lstlisting}[basicstyle=\scriptsize\ttfamily]
% Seed demand with filter values
demand_users_BBB(25, 40, "Smith").

% Magic predicate to guard computation
m_users_BBB(ROOT) :- demand_users_BBB(25, 40, "Smith").

% Guard rule with filter values passed in
users_result(ROOT) :- m_users_BBB(ROOT), users_ext(ROOT), 
                      MINAGE = 25, MAXAGE = 40, NAMECONTAINS = "Smith".
\end{lstlisting}
\end{frame}

\begin{frame}{Effect of Demand Transformation}
  \begin{itemize}
    \item \textbf{Before:} computes up to $|V|^2$ pairs.
    \item \textbf{After:} touches only edges from \texttt{alice} and neighbors.
    \item Metaphor: “Ask who’s hungry, then bake just those slices.”
  \end{itemize}
\end{frame}

\begin{frame}[fragile]{Nested Relationship Example}
\begin{lstlisting}[basicstyle=\scriptsize\ttfamily]
% For a query like: user(id: "1") { posts { comments { author } } }

% Seed initial demand
demand_user_B("1").

% Define magic predicates
m_user_B(UserID) :- demand_user_B(UserID).

% Propagate demand to posts
demand_posts_B(UserID) :- m_user_B(UserID).
m_posts_B(PostID) :- demand_posts_B(UserID), user_posts(UserID, PostID).

% Propagate demand to comments
demand_comments_B(PostID) :- m_posts_B(PostID).
m_comments_B(CommentID) :- demand_comments_B(PostID), post_comments(PostID, CommentID).
\end{lstlisting}
\end{frame}

\begin{frame}{Performance Impact of Demand Transformation}
  \begin{itemize}
    \item \textbf{Without optimization:}
      \begin{itemize}
        \item Processes all users, all posts, all comments
        \item Potentially examines millions of irrelevant records
        \item Filters applied only after full computation
      \end{itemize}
    \item \textbf{With optimization:}
      \begin{itemize}
        \item Starts with specific user "1"
        \item Processes only posts from that user
        \item Retrieves only comments on those posts
        \item Computation proportional to output size, not input size
      \end{itemize}
  \end{itemize}
  
  \begin{block}{Benchmarking Results}
    \item Orders of magnitude performance improvement on large datasets
    \item Higher impact with more selective filters
    \item Critical for complex nested queries in production environments
  \end{block}
\end{frame}


% 11 — Evaluation -----------------------------------------------------
\section{Evaluation}
\begin{frame}{QueryBridge Implementation}
  \begin{itemize}
    \item Python package with modular architecture:
    \begin{itemize}
      \item \texttt{SchemaType} and \texttt{QueryField} data structures
      \item Schema and query parsers using GraphQL-core
      \item Predicate rule generators for XSB Datalog
      \item Demand transformation optimization engine
    \end{itemize}
    \item Comprehensive test suite in \texttt{/tests}:
    \begin{itemize}
      \item Basic schema and query tests
      \item Nested relationship tests
      \item Complex path finding with deep graphs
      \item Variable capitalization tests
    \end{itemize}
    \item Command-line interface: \texttt{python -m querybridge}
    \item Library API for integration
  \end{itemize}
\end{frame}

\begin{frame}{Evaluation: Test Cases and Performance}
  \begin{columns}[T]
    \begin{column}{0.48\textwidth}
      \textbf{Test Cases}
      \begin{itemize}
        \item Basic: Simple queries with 1-2 levels
        \item Nested: Relationships up to 3 levels deep
        \item Path Finding: Graph traversal with complex filters
      \end{itemize}
    \end{column}
    \begin{column}{0.48\textwidth}
      \textbf{Theoretical Performance Improvements}
      \begin{itemize}
            \item Very complex path searching
            \item Large data sets (> 10,000 facts)
            \item Work in Progress
      \end{itemize}
    \end{column}
  \end{columns}
  
  \begin{block}{Key Observations}
    \item Demand transformation benefits increase with query depth
    \item Selective filters show greatest performance gains
    \item Optimized queries scale better with dataset size
  \end{block}
\end{frame}

% 12 — Discussion -----------------------------------------------------
\section{Discussion}
\begin{frame}{Discussion: Trade-offs}
  \begin{block}{Pros}
    \begin{itemize}
      \item Declarative, optimizable intermediate representation
      \item Centralized optimization techniques for any GraphQL implementation
      \item Compatible with existing GraphQL schemas
    \end{itemize}
  \end{block}
  \begin{block}{Cons}
    \begin{itemize}
      \item Initial translation overhead
      \item Additional complexity in the stack
      \item Requires XSB Datalog runtime
      \item Integration challenges with existing systems
    \end{itemize}
  \end{block}
\end{frame}

% 13 — Conclusion -----------------------------------------------------
\section{Conclusion}
\begin{frame}{Conclusion and Future Work}
  \begin{block}{Output}
    \begin{itemize}
      \item Successful translation of GraphQL to XSB Datalog
      \item Efficient demand transformation for query optimization
      \item Support for complex nested queries with deep paths
    \end{itemize}
  \end{block}
  
  \begin{block}{Future Directions}
    \begin{itemize}
      \item Improve performance to match popular graphql implementations (sqlalchemy, hasura)
      \item Integration with window functions
      \item Optimization for non-normalized data via materialized views
    \end{itemize}
  \end{block}
  
  \vspace{0.3cm}
  \centering GitHub: \texttt{github.com/abishekaditya/QueryBridge}
  \bigskip
  \\
  Questions?
\end{frame}

% --------------------------------------------------------------------
\end{document}